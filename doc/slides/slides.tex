\documentclass{beamer}
\beamertemplatenavigationsymbolsempty
\usepackage{pdfpages}
\usepackage{listings}
\usepackage{verbatim}
\usepackage{amssymb}

% todo notes
\usepackage{todonotes}
\presetkeys{todonotes}{inline}{}

% tables
\usepackage{tabu}

% tikz
\usepackage{tikz}
\usetikzlibrary{automata}
\usetikzlibrary{backgrounds}

\tikzstyle{tbox} = [
  draw=fg,
  very thick,
  rectangle,
  rounded corners,
  inner sep=10pt,
  inner ysep=10pt
]
\tikzstyle{tboxtitle} = [rounded corners, fill=fg, text=bg]

% local definitions
\usepackage{environ}

% textbox environment - displays enclosed content inside a box
% @param - box title
\NewEnviron{textbox}[1] {%
  \begin{center}
    \hspace*{-0.5cm}
    \begin{tikzpicture}
      \usebeamercolor[fg]{frametitle}
      \node[tbox] (box) {%
        \begin{minipage}{\textwidth}
          \vspace{0.2cm}
          \BODY
        \end{minipage}};%
      \ifx\relax#1\relax
      \else
        \node[tboxtitle,right=10pt] at (box.north west) {#1};
      \fi
    \end{tikzpicture}
  \end{center}
}

% header text - large, bold, title color
\newcommand{\header}[1]{{\usebeamercolor[fg]{frametitle}{\Large #1}}}

% lst language
\lstdefinestyle{asm}{
  % belowcaptionskip=0,
  morekeywords={[1]
    LOAD,STORE,FENCE,FLUSH,
    ADD,ADDI,SUB,SUBI,MUL,MULI,
    CMP,JMP,JZ,JNZ,JS,JNS,JNZNS,
    MEM,CAS,
    CHECK,HALT,EXIT
  },
  morecomment=[l]\#,
  basicstyle=\footnotesize\ttfamily,
  keywordstyle=\color{blue},
  commentstyle=\color{gray},
}
\lstdefinestyle{c++}{
  belowcaptionskip=1\baselineskip,
  breaklines=true,
  language=C++,
  showstringspaces=false,
  alsoletter={:},
  basicstyle=\scriptsize\ttfamily,
  keywordstyle=\sffamily\bfseries\color{blue},
  commentstyle=\itshape\color{gray},
  stringstyle=\color{red!60!black},
}
\lstdefinestyle{smtlib}{
  breaklines=true,
  morekeywords={[1]
    declare-fun,
    assert,
    not,
    and,
    or,
    xor,
    =,
    =>,
    ite,
    bvadd,
    bvsub,
    bvmul,
    store,
    select,
    extract,
    push,
    pop
  },
  morekeywords={[2]
    BitVec
  },
  alsoletter={-, =, >},
  morecomment=[l]\;,
  sensitive=false,
  basicstyle=\small\ttfamily,
  commentstyle=\color{gray},
  keywordstyle=[1]\bfseries\color{green!40!black},
  keywordstyle=[2]\bfseries\color{red!40!black},
}
\lstdefinestyle{btor2}{
  morekeywords={[1]
    sort,
    zero,
    one,
    constd,
    write,
    input,
    and,
    or,
    nand,
    implies,
    ite,
    add,
    read,
    eq,
    neq,
  },
  morekeywords={[2]
    array,
    bitvec
  },
  morekeywords={[3]
    state,
    init,
    next,
    constraint,
    bad,
  },
  morecomment=[l]\#,
  basicstyle=\small\ttfamily,
  keywordstyle=[1]\bfseries\color{green!40!black},
  keywordstyle=[2]\bfseries\color{blue!80!black},
  keywordstyle=[3]\bfseries\color{red!40!black},
  commentstyle=\color{gray},
}

% check marks
\newcommand{\cmark}[1][green]{{\color{#1} \checkmark}}

% inline code
\newcommand{\lstASM}[1]{\lstinline[style=asm]{#1}}
\newcommand{\lstCPP}[1]{\lstinline[style=c++]{#1}}
\newcommand{\lstSMTLIB}[1]{\lstinline[style=smtlib]{#1}}
\newcommand{\lstBTOR}[1]{\lstinline[style=btor2]{#1}}

% content %%%%%%%%%%%%%%%%%%%%%%%%%%%%%%%%%%%%%%%%%%%%%%%%%%%%%%%%%%%%%%%%%%%%%%
\title{$\textbf{Concu}_{\text{rrent}}\textbf{Bin}_{\text{ary}}\textbf{E}_{\text{valuator}}$\\~\\}
\subtitle{Bounded Model Checking of Lockless Programs}
\author{Florian Schr\"ogendorfer\\~\\\tiny florian.schroegendorfer@phlo.at}
\date{}

\begin{document}

\small

%%%%%%%%%%%%%%%%%%%%%%%%%%%%%%%%%%%%%%%%%%%%%%%%%%%%%%%%%%%%%%%%%%%%%%%%%%%%%%%%
\frame{\titlepage}

%%%%%%%%%%%%%%%%%%%%%%%%%%%%%%%%%%%%%%%%%%%%%%%%%%%%%%%%%%%%%%%%%%%%%%%%%%%%%%%%
\begin{frame}
  \frametitle{Introduction}
  \vfill
  \textbf{Problem}
  \begin{itemize}
    \item heisenbugs due to architecture specific memory ordering habits
    \item not easily testable and impossible to debug
    \item especially troublesome for lockless data structures
    \item requires knowledge about the underlying hardware's characteristics
  \end{itemize}

  \begin{textbox}{Goal}
    Proofing correctness of concurrent programs operating on shared memory.
  \end{textbox}

  \vfill

  \textbf{Approach}
  \begin{itemize}
    \item define a simple machine model as an idealized environment
    \item encode execution of arbitrary programs into SMT formul\ae
    \item proof properties by the means of bounded model checking
  \end{itemize}
\end{frame}

%%%%%%%%%%%%%%%%%%%%%%%%%%%%%%%%%%%%%%%%%%%%%%%%%%%%%%%%%%%%%%%%%%%%%%%%%%%%%%%%
\begin{frame}
  \frametitle{Memory Ordering}

  \vspace{-0.5cm}
  \begin{textbox}{Memory Ordering}
    Order of accesses to memory by a CPU at runtime.
  \end{textbox}

  \textbf{Why reorder memory operations? Performance!}
  \begin{itemize}
    \item CPUs became so fast that caches simply can't keep up
    \item unnoticeable by sequential (single threaded) programs
    \item potentially problematic for parallel programs
  \end{itemize}

  \vfill
  \centering
  \scalebox{0.8}{
    \tabulinesep=4pt
    \begin{tabu}{|l|c|c|c|c|c|c|c|}
      \tabucline{2-}
      \multicolumn{1}{c|}{}
      & \rotatebox{90}{Alpha}
      & \rotatebox{90}{ARM}
      & \rotatebox{90}{Itanium}
      & \rotatebox{90}{MIPS}
      & \rotatebox{90}{POWER}
      & \rotatebox{90}{x86}
      & \rotatebox{90}{zSystems} \\
      \tabucline{2-}
      \firsthline
      Loads Reordered after Loads/Stores? & \cmark & \cmark & \cmark  & \cmark & \cmark & & \\
      \hline
      Stores Reordered after Stores? & \cmark & \cmark & \cmark & \cmark & \cmark & & \\
      \hline
      Stores Reordered after Loads? & \cmark & \cmark & \cmark & \cmark & \cmark & \cmark & \cmark \\
      \hline
      Atomic Reordered with Loads/Stores? & \cmark & \cmark & & \cmark & \cmark & & \\
      \lasthline
    \end{tabu}
  }
\end{frame}

%%%%%%%%%%%%%%%%%%%%%%%%%%%%%%%%%%%%%%%%%%%%%%%%%%%%%%%%%%%%%%%%%%%%%%%%%%%%%%%%
\begin{frame}
  \frametitle{Memory Ordering Models}

  \begin{textbox}{Memory Ordering Model}
    Behaviour of multiprocessor systems regarding memory operations.
  \end{textbox}

  \begin{itemize}
    \item contract between programmer and system
    \item none of the current major architectures is \emph{sequentially consistent}
    \item imposes the requirement for explicit memory barrier instructions
    \item often specified in an informal prose together with litmus tests
  \end{itemize}

  \begin{textbox}{Sequential Consistency}
    \scriptsize
    A multiprocessor is sequentially consistent if the result of any execution is the same as if the operations of all the processors were executed in some sequential order, and the operations of each individual processor appear in this sequence in the order specified by its program. \tiny [Lamport '79]
  \end{textbox}
\end{frame}

%%%%%%%%%%%%%%%%%%%%%%%%%%%%%%%%%%%%%%%%%%%%%%%%%%%%%%%%%%%%%%%%%%%%%%%%%%%%%%%%
\begin{frame}
  \frametitle{Memory Reordering -- Store Buffers}
  \centering
  \hspace{-.4cm}
  \scalebox{.7}{% styles
\tikzstyle{box} = [draw, text centered, rounded corners]
\tikzstyle{processor} = [box, fill=blue!10]
\tikzstyle{buffer} = [box, fill=red!10]
\tikzstyle{cache} = [box, fill=yellow!10]

\begin{tikzpicture}

  % \draw[step=1cm,gray,very thin] (-5, 0) grid (5, -10);

  % processors
  \path [processor] (-5, 0) rectangle (-1, -2);
  \path (-3, -1) node (p-0) [] {\textbf{Processor 0}};

  \path [processor] (1, 0) rectangle (5, -2);
  \path (3, -1) node (p-n) [] {\textbf{Processor n}};

  % store buffers
  \path [buffer] (-3, -3) rectangle (-1, -5);
  \path (-2, -3.7) node (s-0) [] {\textbf{Store}};
  \path (-2, -4.3) node (b-0) [] {\textbf{Buffer}};

  \path [buffer] (1, -3) rectangle (3, -5);
  \path (2, -3.7) node (s-n) [] {\textbf{Store}};
  \path (2, -4.3) node (b-n) [] {\textbf{Buffer}};

  % caches
  \path [cache] (-5, -6) rectangle (-1, -8);
  \path (-3, -7) node (c-0) [] {\textbf{Cache}};

  \path [cache] (1, -6) rectangle (5, -8);
  \path (3, -7) node (c-n) [] {\textbf{Cache}};

  % heap
  \path [box] (-5, -9) rectangle (5, -11);
  \path (0, -10) node (heap) [] {\textbf{Shared Memory}};

  % arrows
  \path [draw, <-] (-4, -2) -- (-4, -6);  % cache -> processor 0
  \path [draw, <->] (-2, -2) -- (-2, -3); % sb <-> processor 0
  \path [draw, ->] (-2, -5) -- (-2, -6);  % sb -> cache
  \path [draw, <->] (-3, -8) -- (-3, -9); % cache <-> memory

  \path [draw, <-] (4, -2) -- (4, -6);    % cache -> processor n
  \path [draw, <->] (2, -2) -- (2, -3);   % sb -> processor n
  \path [draw, ->] (2, -5) -- (2, -6);    % sb -> cache
  \path [draw, <->] (3, -8) -- (3, -9);   % cache <-> memory

  \path [draw, <-] (-1, -7) -- (-0.5, -7);
  \path (0, -7) node (dots) [] {\large $\ldots$};
  \path [draw, ->] (0.5, -7) -- (1, -7);

  % dots
  \path (0, -1) node (dots) [] {\large $\ldots$};

\end{tikzpicture}
}
\end{frame}

%%%%%%%%%%%%%%%%%%%%%%%%%%%%%%%%%%%%%%%%%%%%%%%%%%%%%%%%%%%%%%%%%%%%%%%%%%%%%%%%
\begin{frame}[fragile]
  \frametitle{Memory Reordering -- Store Buffer Litmus Test}
  \begin{lstlisting}[style=c++, numbers=left, numberstyle=\tiny\color{gray}, numberblanklines=false]
static int w0 = 0, w1 = 0;
static int r0 = 0, r1 = 0;

static void * T0 (void * t) {
  w0 = 1;
  r0 = w1;
  return t;
}

static void * T1 (void * t) {
  w1 = 1;
  r1 = w0;
  return t;
}

int main () {
  pthread_t t[2];
  pthread_create(t + 0, 0, T0, 0);
  pthread_create(t + 1, 0, T1, 0);
  pthread_join(t[0], 0);
  pthread_join(t[1], 0);
  assert(r0 + r1);
  return 0;
}
  \end{lstlisting}
\end{frame}

%%%%%%%%%%%%%%%%%%%%%%%%%%%%%%%%%%%%%%%%%%%%%%%%%%%%%%%%%%%%%%%%%%%%%%%%%%%%%%%%
\begin{frame}
  \frametitle{Memory Reordering -- Store Buffer Litmus Test Trace}
  \centering
  \begin{tikzpicture}

  \path (0, 0) node (init) [] {$w_0 = w_1 = 0$};
  \path (0, -6) node (join) [] {$r_0 + r_1 = 0$};

  \path (-3, -2) node (w0) [red] {$w_0 = 1$};
  \path [->] (init) edge node [above] {$t_0$} (w0);
  \path (-3, -4) node (r0) [] {$r_0 = w_1 = 0$};
  \path [->] (w0) edge node [left] {$t_0$} (r0);
  \path [->] (r0) edge node [below] {$t_0$} (join);

  \path (3, -2) node (w1) [red] {$w_1 = 1$};
  \path [->] (init) edge node [above] {$t_1$} (w1);
  \path (3, -4) node (r1) [] {$r_1 = w_0 = 0$};
  \path [->] (w1) edge node [right] {$t_1$} (r1);
  \path [->] (r1) edge node [below] {$t_1$} (join);

  \path (0, -2) node (buffered) [red] {buffered};
  \path [draw, dotted, red, <-] (w0) -- (buffered);
  \path [draw, dotted, red, ->] (buffered) -- (w1);

  \path [draw, dotted, ->] (init) -- (r0);
  \path [draw, dotted, ->] (init) -- (r1);

  % forwarding arrows

  % \path [draw, dotted, green, ->] (init) edge node [near end, left] {\cmark} (r0);
  % \path [draw, dotted, green, ->] (init) edge node [near end, right] {$\;$\cmark} (r1);

  % \path [dotted, red, ->] (w1) edge node [near end, below] {\xmark} (r0);
  % \path [dotted, red, ->] (w0) edge node [near end, below] {\xmark} (r1);

  % \path (0, -1.2) node (mem) [green] {memory};
  % \path (0, -3.6) node (sb) [red] {buffered};

  % alternative figure

  % \path (0, -10) node (Xinit) [draw, rounded corners, text centered, inner sep=0pt] {
    % \begin{tabu}{c}
      % \footnotesize memory \\
      % $w_0 = w_1 = r_0 = r_1 = 0$ \\
    % \end{tabu}
  % };
%
  % \path (-3, -13) node (Xw0) [draw, rounded corners, text centered, inner sep=0pt] {
    % \begin{tabu}{c}
      % \textbf{thread 0} \\
      % \hline
      % \footnotesize memory \\
      % $w_1 = r_0 = r_1 = 0$ \\
      % % \tabucline[on 1pt off 2pt]{-}
      % \hline
      % \footnotesize store buffer \\
      % % \tabucline[on 1pt off 2pt]{-}
      % $w_0 = 1$
    % \end{tabu}
  % };
%
  % \path (-3, -17) node (Xr0) [draw, rounded corners, text centered, inner sep=0pt] {
    % \begin{tabu}{c}
      % \textbf{thread 0} \\
      % \hline
      % \footnotesize memory \\
      % $w_1 = r_0 = r_1 = 0$ \\
      % % \tabucline[on 1pt off 2pt]{-}
      % \hline
      % \footnotesize store buffer \\
      % % \tabucline[on 1pt off 2pt]{-}
      % $w_0 = 1\;$
    % \end{tabu}
  % };
%
  % \path [->] (Xinit) edge node [left] {$w_0 = 1$} (Xw0);
  % \path [->] (Xw0) edge node [left] {$r_0 = w_1$} (Xr0);
\end{tikzpicture}

\end{frame}

%%%%%%%%%%%%%%%%%%%%%%%%%%%%%%%%%%%%%%%%%%%%%%%%%%%%%%%%%%%%%%%%%%%%%%%%%%%%%%%%
\begin{frame}
  \frametitle{Virtual Machine Model -- Architecture}
  \centering
  \hspace{-.8cm}
  \scalebox{.8}{% layers
\pgfdeclarelayer{background}
\pgfdeclarelayer{foreground}
\pgfsetlayers{background, main, foreground}

% styles
\tikzstyle{box} = [draw, text centered, rounded corners]
\tikzstyle{register} = [box, fill=white, minimum height=2em, minimum width=4em]
\tikzstyle{processor} = [draw, fill=blue!10, rounded corners]
\tikzstyle{buffer} = [processor, fill=red!10, dashed]

% distances
\def\blockdist{3}
\def\borderdist{0.2}

\begin{tikzpicture}

  % processor 0 %%%%%%%%%%%%%%%%%%%%%%%%%%%%%%%%%%%%%%%%%%%%%%%%%%%%%%%%%%%%%%%%
  \path (0, 0) node (title-0) [] {\textbf{Processor 0}};

  % store buffer
  \path (title-0)+(0, -1) node (sb-0) [] {store buffer};
  \path (sb-0)+(0, -1) node (full-0) [register] {full};
  \path (full-0)+(0, -1) node (adr-0) [register] {address};
  \path (adr-0)+(0, -1) node (val-0) [register] {value};

  % registers
  \path (val-0)+(-1.1, -1.5) node (accu-0) [register] {accu};
  \path (val-0)+(1.1, -1.5) node (mem-0) [register, dashed] {mem};

  % paths
  \path [draw, <->] (val-0.west) -- (val-0.west -| accu-0.north) -- (accu-0.north);
  \path [draw, ->] (val-0.east) -- (val-0.east -| mem-0.north) -- (mem-0.north);

  \begin{pgfonlayer}{background}
    % bounding box
    \path (accu-0.west |- title-0.north)+(-\borderdist, \borderdist) node (a) {};
    \path (mem-0.south east)+(\borderdist, -\borderdist) node (b) {};
    \path [processor] (a) rectangle (b);

    % top left heap coordinate
    \path (a |- b)+(0, -1) node (tlh) {};

    % store buffer box
    \path (sb-0.north west)+(-\borderdist, \borderdist) node (a) {};
    \path (val-0.south -| sb-0.east)+(\borderdist, -\borderdist) node (b) {};
    \path [buffer] (a) rectangle (b);
  \end{pgfonlayer}

  % dots %%%%%%%%%%%%%%%%%%%%%%%%%%%%%%%%%%%%%%%%%%%%%%%%%%%%%%%%%%%%%%%%%%%%%%%
  \path (title-0)+(\blockdist, 0) node (dots) [] {\large $\ldots$};

  % processor n %%%%%%%%%%%%%%%%%%%%%%%%%%%%%%%%%%%%%%%%%%%%%%%%%%%%%%%%%%%%%%%%
  \path (dots)+(\blockdist, 0) node (title-n) [] {\textbf{Processor n}};

  % store buffer
  \path (title-n)+(0, -1) node (sb-n) [] {store buffer};
  \path (sb-n)+(0, -1) node (full-n) [register] {full};
  \path (full-n)+(0, -1) node (adr-n) [register] {address};
  \path (adr-n)+(0, -1) node (val-n) [register] {value};

  % registers
  \path (val-n)+(-1.1, -1.5) node (accu-n) [register] {accu};
  \path (val-n)+(1.1, -1.5) node (mem-n) [register, dashed] {mem};

  % paths
  \path [draw, <->] (val-n.west) -- (val-n.west -| accu-n.north) -- (accu-n.north);
  \path [draw, ->] (val-n.east) -- (val-n.east -| mem-n.north) -- (mem-n.north);

  \begin{pgfonlayer}{background}
    % bounding box
    \path (accu-n.west |- title-n.north)+(-\borderdist, \borderdist) node (a) {};
    \path (mem-n.south east)+(\borderdist, -\borderdist) node (b) {};
    \path [processor] (a) rectangle (b);

    % bottom right heap coordinate
    \path (b)+(0, -\blockdist) node (brh) {};

    % store buffer box
    \path (sb-n.north west)+(-\borderdist, \borderdist) node (a) {};
    \path (val-n.south -| sb-n.east)+(\borderdist, -\borderdist) node (b) {};
    \path [buffer] (a) rectangle (b);
  \end{pgfonlayer}

  % heap %%%%%%%%%%%%%%%%%%%%%%%%%%%%%%%%%%%%%%%%%%%%%%%%%%%%%%%%%%%%%%%%%%%%%%%
  \path [box] (tlh) rectangle node (heap) [minimum height=2cm] {shared memory} (brh);

  % paths
  \path [draw, ->] (val-0) -- (heap.north -| val-0);
  \path [draw, <->] (accu-0) -- (heap.north -| accu-0);
  \path [draw, <-] (mem-0) -- (heap.north -| mem-0);

  \path [draw, ->] (val-n) -- (heap.north -| val-n);
  \path [draw, <->] (accu-n) -- (heap.north -| accu-n);
  \path [draw, <-] (mem-n) -- (heap.north -| mem-n);

\end{tikzpicture}
}
\end{frame}

%%%%%%%%%%%%%%%%%%%%%%%%%%%%%%%%%%%%%%%%%%%%%%%%%%%%%%%%%%%%%%%%%%%%%%%%%%%%%%%%
\begin{frame}[fragile]
  \frametitle{Virtual Machine Model -- Instruction Set}
  \vspace*{-.2cm}
  \begin{columns}
    \begin{column}{0.2\textwidth}
      \begin{textbox}{Memory}\color{black}
        \begin{tabular}{ll}
          \lstASM{LOAD} & \lstASM{adr} \\
          \lstASM{STORE} & \lstASM{adr} \\
          \lstASM{FENCE} \\
        \end{tabular}
      \end{textbox}
      \begin{textbox}{Arithmetic}\color{black}
        \begin{tabular}{ll}
          \lstASM{ADD} & \lstASM{adr} \\
          \lstASM{ADDI} & \lstASM{val} \\
          \lstASM{SUB} & \lstASM{adr} \\
          \lstASM{SUBI} & \lstASM{val} \\
          \lstASM{MUL} & \lstASM{adr} \\
          \lstASM{MULI} & \lstASM{val} \\
        \end{tabular}
      \end{textbox}
    \end{column}

    \begin{column}{0.2\textwidth}
      \begin{textbox}{Atomic}\color{black}
        \begin{tabular}{ll}
          \lstASM{MEM} & \lstASM{adr} \\
          \lstASM{CAS} & \lstASM{adr} \\
        \end{tabular}
      \end{textbox}
      \vspace{0.1cm}
      \begin{textbox}{Flow Control}\color{black}
        \begin{tabular}{ll}
          \lstASM{CMP} & \lstASM{adr} \\
          \lstASM{JZ} & \lstASM{pc} \\
          \lstASM{JNZ} & \lstASM{pc} \\
          \lstASM{JS} & \lstASM{pc} \\
          \lstASM{JNS} & \lstASM{pc} \\
          \lstASM{JNZNS} & \lstASM{pc} \\
        \end{tabular}
      \end{textbox}
    \end{column}

    \begin{column}{0.2\textwidth}
      \vspace{-2.1cm}
      \begin{textbox}{Termination}\color{black}
        \begin{tabular}{ll}
          \lstASM{HALT} \\
          \lstASM{EXIT} & \lstASM{val} \\
        \end{tabular}
      \end{textbox}
      \vspace{0.1cm}
      \begin{textbox}{Meta}\color{black}
        \begin{tabular}{ll}
          \lstASM{CHECK} & \lstASM{id} \\
        \end{tabular}
      \end{textbox}
    \end{column}
  \end{columns}
  \vfill
\end{frame}

%%%%%%%%%%%%%%%%%%%%%%%%%%%%%%%%%%%%%%%%%%%%%%%%%%%%%%%%%%%%%%%%%%%%%%%%%%%%%%%%
\begin{frame}
  \frametitle{ConcuBinE}
  \vspace{-.5cm}
  \begin{center}
    \scalebox{.8}{% styles
\tikzstyle{module} = [draw, fill=blue!10, text centered, rounded corners, minimum height=1cm, minimum width=2.5cm]
\tikzstyle{file} = [module, fill=red!10]

\begin{tikzpicture}[auto]

  \node [file] (program) {program};

  \node [module] (simulate) [above right=0cm and 1cm of program] {\texttt{simulate}};
  \path [draw, ->] (program.east) -- ++(0.5, 0) |- (simulate);

  \node [module] (solve) [below right=0cm and 1cm of program] {\texttt{solve}};
  \path [draw, ->] (program.east) -- ++(0.5, 0) |- (solve);

  \node [file] (trace) [below right=0cm and 1cm of simulate] {trace};
  \path [draw, ->] (simulate.east) -- ++(0.5, 0) |- (trace);
  \path [draw, ->] (solve.east) -- ++(0.5, 0) |- (trace);

  \node [module, dashed] (replay) [right= of trace] {\texttt{replay}} edge [<-, dashed] (trace);
  \path [draw, dashed, ->] (replay) -- (replay.north |- simulate) -| (trace.north);

\end{tikzpicture}
}
  \end{center}
  \begin{itemize}
    \item \texttt{simulate} -- simulate execution
    \item \texttt{solve} -- encode and solve bounded model checking problem
      \begin{itemize}
        \item generates SMT-LIB or Btor2 encodings
        \item checks for an exit code greater than zero by default
        \item uses Boolector (BtorMC), Z3 and CVC4 as backend solvers
        \item returs erroneous trace if problem was indeed satisfiable
      \end{itemize}
    \item \texttt{replay} -- reevaluate trace via simulation
  \end{itemize}
\end{frame}

%%%%%%%%%%%%%%%%%%%%%%%%%%%%%%%%%%%%%%%%%%%%%%%%%%%%%%%%%%%%%%%%%%%%%%%%%%%%%%%%
\begin{frame}[fragile]
  \frametitle{ConcuBinE -- Store Buffer Litmus Test}
  \begin{tikzpicture}
    \usebeamercolor[fg]{frametitle}

    % mmap
    % \node at (0, -3.2) {\color{red}+};
    \path (0, -3.2) node (mmap) [tbox, text width=.83cm, align=center] {
        \color{black}
        \vspace{-.05cm}\\
        % \begin{tabular}{ll}
          % \texttt{0} & \texttt{0} \\
          % \texttt{1} & \texttt{0}
        % \end{tabular}
        \texttt{0 0} \\
        \texttt{1 0}
    };
    \node [tboxtitle,right=5pt] at (mmap.north west) {MMap};

    % checker
    % \node at (0, 0) {\color{red}+};
    \path (0, 0) node (chk) [tbox] {
        \color{black}
        \begin{tabular}{ll}
          \lstASM{CHECK} & \lstASM{0} \\
          \lstASM{ADD} & \lstASM{0} \\
          \lstASM{ADD} & \lstASM{1} \\
          \lstASM{JZ} & \lstASM{error} \\
          \lstASM{EXIT} & \lstASM{0} \\
          \lstASM{error: EXIT} & \lstASM{1} \\
        \end{tabular}
    };
    \node[tboxtitle,right=10pt] at (chk.north west) {Checker};

    % thread 1
    % \node at (4, 0) {\color{red}+};
    \path (4, 0) node (t1) [tbox] {
        \color{black}
        \begin{tabular}{ll}
          \lstASM{ADDI} & \lstASM{1} \\
          \lstASM{STORE} & \lstASM{0} \\
          \lstASM{LOAD} & \lstASM{1} \\
          \lstASM{CHECK} & \lstASM{0} \\
        \end{tabular}
    };
    \node[tboxtitle,right=10pt] at (t1.north west) {Thread 1};

    % thread 2
    % \node at (7.2, 0) {\color{red}+};
    \path (7.2, 0) node (t2) [tbox] {
        \color{black}
        \begin{tabular}{ll}
          \lstASM{ADDI} & \lstASM{1} \\
          \lstASM{STORE} & \lstASM{1} \\
          \lstASM{LOAD} & \lstASM{0} \\
          \lstASM{CHECK} & \lstASM{0} \\
        \end{tabular}
    };
    \node[tboxtitle,right=10pt] at (t2.north west) {Thread 2};

    % command
    % \node at (3, -5) {\color{red}+};
    \path (3, -5) node (cmd) [] {
      \footnotesize
      \color{black}
      \texttt{\$ concubine solve -m init.mmap 15 checker.asm t1.asm t2.asm}
    };

    % mmap arrow
    % \node at (2.3, -4.8) {\color{green}+};
    \path [draw, -stealth] (mmap) -- (2.1, -4.7);
    % checker arrow
    % \node at (4.6, -4.8) {\color{green}+};
    \path [draw, -stealth] (chk) -- (4.5, -4.7);
    % thread 1 arrow
    % \node at (6.2, -4.8) {\color{green}+};
    \path [draw, -stealth] (t1) -- (6.1, -4.7);
    % thread 2 arrow
    % \node at (7.4, -4.8) {\color{green}+};
    \path [draw, -stealth] (t2) -- (7.3, -4.7);
  \end{tikzpicture}
\end{frame}

%%%%%%%%%%%%%%%%%%%%%%%%%%%%%%%%%%%%%%%%%%%%%%%%%%%%%%%%%%%%%%%%%%%%%%%%%%%%%%%%
\begin{frame}[fragile]
  \frametitle{ConcuBinE -- Store Buffer Litmus Test Trace}
  \centering
  \begin{lstlisting}[style=asm,xleftmargin=-.3cm]
checker.asm
t1.asm
t2.asm
. smt.mmap
# tid pc    cmd   arg   accu  mem adr val full  heap
1     0     ADDI  1     0     0   0   0   0     {}  # 0
2     0     ADDI  1     0     0   0   0   0     {}  # 1
1     1     STORE 0     1     0   0   0   0     {}  # 2
2     1     STORE 1     1     0   0   0   0     {}  # 3
1     2     LOAD  1     1     0   0   1   1     {}  # 4
2     2     LOAD  0     1     0   1   1   1     {}  # 5
1     3     CHECK 0     0     0   0   1   1     {}  # 6
2     3     CHECK 0     0     0   1   1   1     {}  # 7
0     0     CHECK 0     0     0   0   0   0     {}  # 8
0     1     ADD   0     0     0   0   0   0     {}  # 9
0     2     ADD   1     0     0   0   0   0     {}  # 10
0     3     JZ    error 0     0   0   0   0     {}  # 11
0     error EXIT  1     0     0   0   0   0     {}  # 12
\end{lstlisting}
\end{frame}

\end{document}
