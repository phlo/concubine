\todo[inline]{\url{https://arxiv.org/pdf/cs/0108016.pdf}}
\todo[inline]{\url{https://www.cl.cam.ac.uk/~pes20/weakmemory/cacm.pdf}}
\todo[inline]{\url{https://mirrors.edge.kernel.org/pub/linux/kernel/people/paulmck/perfbook/perfbook-e2-rc9.pdf}}
% CBMC
\todo[inline]{\url{http://www.kroening.com/papers/tcad-sw-2008.pdf}}
\todo[inline]{\url{http://www.kroening.com/papers/tacas2004.pdf}}
\todo[inline]{\url{https://link.springer.com/content/pdf/10.1007/978-3-642-39799-8\_9.pdf}}

\newpage

\section{Introduction}

%% sequential consistency:
%% \emph{``the result of any execution is the same as if the operations of all the processors were executed in some sequential order, and the operations of each individual processor appear in this sequence in the order specified by its program''}.

% \todo[inline]{need for software verification}\noindent

Correctness of computer systems is critical in today's information age and although software verification made considerable progress in the last decade, it is still an ongoing research topic.
Testing alone however is not sufficient to validate parallel programs
communicating via shared memory on modern multiprocessor hardware,
% running on current multiprocessor hardware
% and communicating via shared memory,
% running on modern shared memory multiprocessor hardware
as fatal race conditions can have extremely low probabilities of occurrence.
The situation becomes even worse due to counter-intuitive behaviour introduced by certain hardware optimizations.
Without further knowledge about the underlying architecture, inexperienced developers of
concurrent
low-level
systems code like lock-free data structures, operating system kernels, synchronisation
% libraries,
primitives,
% compilers and so on might assume that the order of access to memory is \emph{sequentially consistent}
% compilers and so on might assume a total order of access to memory for any possible interleaving, where each process is executed in program order.
compilers,
% etc.\
and so on
might assume
% that
% \emph{``the result of any execution is the same as if the operations of all the processors were executed in some sequential order for every possible interleaving and the operations of each individual processor appear in this sequence in the order specified by its program''}
% the result of any execution is the same as if the operations of all processors were executed in some sequential order and the operations of each individual processor appear program order.
\emph{sequential consistency} \cite{ref:Lamport79}.
Unfortunately, none of the major hardware architectures follows this rather restrictive memory ordering model and allow
% memory access operations to be reordered
reordering of memory access operations
in various ways.
This opens the door for hard to find bugs caused by unexpected
% reorderings
behaviour
and
% needs to be prevented by
% requires careful use of the target architecture's memory barrier operations to ensure consistency.
requires a deep understanding of the target architecture's memory ordering model,
% as well as
accompanied by careful use of memory barrier operations to ensure consistency.
% Table \ref{tbl:ordering} shows a rough overview of the memory ordering models used by different architectures.
% referred to as \emph{sequential consistency} \cite{ref:Lamport79}.

\todo[inline]{lack of concrete, formal specifications what programmers can rely on}


% \bigskip

% For example, modern processors are equipped with a store buffer to speed up writes.

% \todo[inline]{counter-intuitive behaviour introduced by hardware optimizations}
% \todo[inline]{memory models}

% \bigskip

\tabulinesep=6pt
\noindent
\begin{table}[!hbt]
  \centering
  \begin{tabu}{|l|c|c|c|c|c|c|c|c|}
    \tabucline{2-}
    \multicolumn{1}{c|}{}
    & \rotatebox{90}{Alpha}
    & \rotatebox{90}{ARM}
    & \rotatebox{90}{Itanium}
    & \rotatebox{90}{MIPS}
    & \rotatebox{90}{POWER}
    & \rotatebox{90}{SPARC-TSO}
    & \rotatebox{90}{x86}
    & \rotatebox{90}{zSystems} \\
    \tabucline{2-}
    \firsthline
    Loads Reordered after Loads/Stores? & \cmark & \cmark & \cmark  & \cmark & \cmark & & & \\
    \hline
    Stores Reordered after Stores? & \cmark & \cmark & \cmark & \cmark & \cmark & & & \\
    \hline
    Stores Reordered after Loads? & \cmark & \cmark & \cmark & \cmark & \cmark & \cmark & \cmark & \cmark \\
    \hline
    Atomic Reordered with Loads/Stores? & \cmark & \cmark & & \cmark & \cmark & & & \\
    % \hline
    % Dependent Loads Reordered? & \cmark & & & & & & & \\
    % \hline
    % Dependent Stores Reordered & & & & & & & & \\
    \lasthline
  \end{tabu}
  \caption{Summary of Memory Ordering}
  \label{tbl:ordering}
\end{table}

% \subsection{Problem/Motivation}

% \todo[inline]{problems arising due to the unawareness of the specific architectures memory model}
% \todo[inline]{example for counter intuitive behaviour} \noindent

\todo[inline]{reordering introduced by store buffers}

Table \ref{tbl:ordering} shows that even the most restrictive architectures like the widely used x86

\bigskip

The Intel-64 and AMD memory ordering models allow a load to be reordered with an earlier store to a different location, thus breaking sequential consistency.
However, loads are not reordered with stores to the same location.
Even though this is the only reordering allowed, it is sufficient for introducing counter intuitive behaviour, illustrated by the example given in Listing \ref{fig:intro:code}.

\newpage

\begin{lstlisting}[style=c++, numbers=left, numberstyle=\footnotesize, numberblanklines=false, caption={Store Buffer Litmus Test}, label={fig:intro:code}]
#include <pthread.h>
#include <assert.h>

#define ACCESS(x) (*(volatile typeof(x) *) &(x))
#define READ(x) ({typeof(x) TMP = ACCESS(x); TMP;})
#define WRITE(x,v) ({ACCESS(x) = (v);})

static int w0 = 0;
static int w1 = 0;

static int r0 = 0;
static int r1 = 0;

static void * P0 (void * p)
{
  WRITE(w0, 1);
  r0 = READ(w1);
  return p;
}

static void * P1 (void * p)
{
  WRITE(w1, 1);
  r1 = READ(w0);
  return p;
}

int main ()
{
  pthread_t t[2];
  pthread_create (t + 0, 0, P0, 0);
  pthread_create (t + 1, 0, P1, 0);
  pthread_join (t[0], 0);
  pthread_join (t[1], 0);
  assert(r0 + r1);
  return 0;
}
\end{lstlisting}

\begin{figure}[!h]
  \centering
  % distances
% \def\blockdist{3}
% \def\borderdist{0.2}
%
% \tabulinesep=0.1cm

\begin{tikzpicture}
  % \path (0,0) node (init) [] {
    % $\begin{aligned}
      % r_0 &= w_0 = 0 \\
      % r_1 &= w_1 = 0
    % \end{aligned}$};

  \path (0, 0) node (init) [] {$w_0 = w_1 = 0$};
  \path (0, -6) node (join) [] {$r_0 + r_1 = 0$};

  \path (-3, -2) node (w0) [red] {$w_0 = 1$};
  \path [->] (init) edge node [above] {$t_0$} (w0);
  \path (-3, -4) node (r0) [] {$r_0 = w_1 = 0$};
  \path [->] (w0) edge node [left] {$t_0$} (r0);
  \path [->] (r0) edge node [below] {$t_0$} (join);

  \path (3, -2) node (w1) [red] {$w_1 = 1$};
  \path [->] (init) edge node [above] {$t_1$} (w1);
  \path (3, -4) node (r1) [] {$r_1 = w_0 = 0$};
  \path [->] (w1) edge node [right] {$t_1$} (r1);
  \path [->] (r1) edge node [below] {$t_1$} (join);

  \path (0, -2) node (buffered) [red] {buffered};
  \path [draw, dotted, red, <-] (w0) -- (buffered);
  \path [draw, dotted, red, ->] (buffered) -- (w1);

  \path [draw, dotted, ->] (init) -- (r0);
  \path [draw, dotted, ->] (init) -- (r1);

  % \path [draw, dotted, green, ->] (init) edge node [near end, left] {\cmark} (r0);
  % \path [draw, dotted, green, ->] (init) edge node [near end, right] {$\;$\cmark} (r1);

  % \path [dotted, red, ->] (w1) edge node [near end, below] {\xmark} (r0);
  % \path [dotted, red, ->] (w0) edge node [near end, below] {\xmark} (r1);

  % \path (0, -1.2) node (mem) [green] {memory};
  % \path (0, -3.6) node (sb) [red] {buffered};

  % alternative

  % \path (0, -10) node (Xinit) [draw, rounded corners, text centered, inner sep=0pt] {
    % \begin{tabu}{c}
      % \footnotesize memory \\
      % $w_0 = w_1 = r_0 = r_1 = 0$ \\
    % \end{tabu}
  % };
%
  % \path (-3, -13) node (Xw0) [draw, rounded corners, text centered, inner sep=0pt] {
    % \begin{tabu}{c}
      % \textbf{thread 0} \\
      % \hline
      % \footnotesize memory \\
      % $w_1 = r_0 = r_1 = 0$ \\
      % % \tabucline[on 1pt off 2pt]{-}
      % \hline
      % \footnotesize store buffer \\
      % % \tabucline[on 1pt off 2pt]{-}
      % $w_0 = 1$
    % \end{tabu}
  % };
%
  % \path (-3, -17) node (Xr0) [draw, rounded corners, text centered, inner sep=0pt] {
    % \begin{tabu}{c}
      % \textbf{thread 0} \\
      % \hline
      % \footnotesize memory \\
      % $w_1 = r_0 = r_1 = 0$ \\
      % % \tabucline[on 1pt off 2pt]{-}
      % \hline
      % \footnotesize store buffer \\
      % % \tabucline[on 1pt off 2pt]{-}
      % $w_0 = 1\;$
    % \end{tabu}
  % };
%
  % \path [->] (Xinit) edge node [left] {$w_0 = 1$} (Xw0);
  % \path [->] (Xw0) edge node [left] {$r_0 = w_1$} (Xr0);
\end{tikzpicture}

  \caption{Store Buffer Litmus Test Trace}
\end{figure}

\todo[inline]{x86 permits $\texttt{r0 = 0} \land \texttt{r1 = 0}$}

\todo[inline]{\texttt{experiments/demo/run.sh}: 177169 out of 1000000 failed = $\sim 17.7$ \% ($\sim 12$ min)}
\todo[inline]{only 82 out of 1000000 returned the perfectly legal outcome of 2}

\subsection{Toolchain}

\begin{figure}[h]
  \centering
  % styles
\tikzstyle{module} = [draw, fill=blue!10, text centered, rounded corners, minimum height=1cm, minimum width=2.5cm]
\tikzstyle{file} = [module, fill=red!10]
\tikzstyle{external} = [module, fill=yellow!10]

\begin{tikzpicture}[auto]

  \node [file] (program) {program};

  \node [module] (simulate) [above right=0cm and 1cm of program] {\texttt{simulate}};
  \path [draw, ->] (program.east) -- ++(0.5, 0) |- (simulate);

  \node [module] (solve) [below right=0cm and 1cm of program] {\texttt{solve}};
  \path [draw, ->] (program.east) -- ++(0.5, 0) |- (solve);

  \node [external] (solver) [below= of solve] {SMT solver};
  \path [draw, ->, transform canvas={xshift=-0.5cm}] (solve) -- (solver);
  \path [draw, <-, transform canvas={xshift=0.5cm}] (solve) -- (solver);

  \node [file] (trace) [below right=0cm and 1cm of simulate] {trace};
  \path [draw, ->] (simulate.east) -- ++(0.5, 0) |- (trace);
  \path [draw, ->] (solve.east) -- ++(0.5, 0) |- (trace);

  \node [module, dashed] (replay) [right= of trace] {\texttt{replay}} edge [<-, dashed] (trace);
  \path [draw, dashed, ->] (replay) -- (replay.north |- simulate) -| (trace.north);

\end{tikzpicture}

  \caption{Toolchain}
\end{figure}

\todo[inline]{components structured according to the mode they are used}
