\section{BTOR2}

\newcommand{\LPREV}{\rotatebox[origin=c]{180}{$\Lsh$}}
\newcommand{\RPREV}{\rotatebox[origin=c]{180}{$\Rsh$}}

% We also included the possibility to generate the novel word level model checking format \BTOR \cite{ref:BTOR2}, providing a sequential extensions for specifying word-level model checking problems with registers and memories.
We also included the possibility to generate the word level model checking format {\BTOR} \cite{ref:BTOR2}, using the \texttt{-e btor2} command line parameter.
% providing a \emph{sequential extension} for specifying % register and array states in combination with a transition function,
% By providing a \emph{sequential extension} for specifying % register and array states in combination with a transition function,
It provides a \emph{sequential extension} for specifying % register and array states in combination with a transition function,
states in combination with their transition functions,
which are automatically unrolled via symbolic substitution by the accompanying bounded model checker BtorMC.
% the generated formulas size remains constant with respect to the upper bound, compared to the linear growth of our {\SMTLIB} encodings.
% it produces constant sized formulas with respect to the upper bound
% In contrast to \SMTLIB, which requires manual unrolling, \BTOR produces constant sized formulas with respect to the upper bound.
% Automatic unrolling via symbolic substitution by the accompanying bounded model checker BtorMC
% >>>
In contrast to the linear growth of our {\SMTLIB} encodings, the generated formulas size therefore remains constant for any given bound. % with respect to the upper bound.
% The generated formulas size therefore remains constant with respect to the upper bound, compared to the linear growth of our {\SMTLIB} encodings.
% In contrast to {\SMTLIB}, which requires manual unrolling, it produces constant sized formulas with respect to the upper bound by providing a \emph{sequential extension} for specifying states in combination with their transition functions.
% states in combination with their transition functions.
% symbolicly substituting current state expressions into next state functions and incremental SMT solving.
% Its line based syntax on the other hand requires more housekeeping, leading to a slightly more complex encoding process.
% We omit an outline of the encoding process, as it is semantically equivalent to the {\SMTLIB} formulas using our functional next state logic and a bit more tedious due to the line based syntax.
% We omit an outline of the encoding process, as it is semantically equivalent to the previous one using our functional next state logic and a bit more tedious due to the line based syntax.
We omit further details about the encoding process, as it is more or less identical to the previous using our functional next state logic, but a bit more tedious due to the line based syntax.
% Instead, the encoding of our store buffer litmus test example is reviewed st
Instead, we review the encoding of our store buffer litmus test example to highlight the pros and cons of {\BTOR}.

% \bigskip
% \noindent
{\BTOR} formulas consist of nodes, one per line, each prefixed with a strictly increasing numeric identifier serving as the input to subsequent nodes.
% > harder to read | decrease readability
% > easier to parse | implicitly cycle free
% > direct access to every part of the expression tree
By requiring every node to be defined before being used, the resulting direct acyclic graph is easy to parse
% and allows transparent access to every part of the expression, thus avoiding the need for additional auxiliary variables.
% and no additional auxiliary variables are needed since every part of the expression may be referenced at a later point.
and since every part of the expression may be referenced at a later point,
% the need for additional auxiliary variables is avoided.
no additional auxiliary variables are needed.
% While this decreases readability, the resulting direct acyclic graph is easy to parse and allow direct reference to any part of without the need of any additional variable.
% These main restrictions slightly decrease readability but easier to pass since everything needs to be defined before
% By using these strictly increasing identifiers as inputs to subsequent nodes, no other
% By requiring these strictly increasing identifiers to be the only form of input for subsequent nodes,
% no other variables are needed to build the expression tree
% These strictly increasing identifiers are used as a reference during the definition of subsequent nodes for
% used as a reference in the definition of subsequent nodes
% Primary inputs to the formula are declared by the \lstBTOR{input} keyword for free variables and \lstBTOR{state} for flags, registers and arrays.
See \cite{ref:BTOR2} for a detailed format description and the list of available nodes.

The encoding of our store buffer litmus test example given in Table \ref{tbl:encoding:programs} starts by defining the required boolean, bit-vector and array sorts.
\begin{lstlisting}[style=btor2]
1 sort bitvec 1   `\lstCMT{$\BVSORT[1]$}`
2 sort bitvec 16  `\lstCMT{$\BVSORT$}`
3 sort array 2 2  `\lstCMT{$\ASORT$}`
\end{lstlisting}
Next, a special set of input nodes is used to define the required boolean
% Because operands are limited to identifiers, the required boolean and
% Followed by input nodes for the required boolean
\begin{lstlisting}[style=btor2]
4 zero 1
5 one 1
\end{lstlisting}
% bit-vector constants have to be defined through a special set of input nodes.
and bit-vector constants.
\begin{lstlisting}[style=btor2]
6 zero 2
7 one 2
8 constd 2 5
9 constd 2 17
\end{lstlisting}

\noindent
% Register and array states are declared by the keyword \lstBTOR{state}, may be initialized using \lstBTOR{init} and their transition function defined via \lstBTOR{next}.
% In order to initialize our $\HEAP$ array according to example's memory map given in Listing \ref{lst:mmap:init.mmap},
% we introduce a \lstBTOR{state} node.
% Initialize temporary memory map array state.
% the keyword \lstBTOR{state} to declare registers and memories in combination with their transition function through \lstBTOR{next}.
% The main advantage of {\BTOR} is the ability to declare states via the keyword \lstBTOR{state} and their transition function using \lstBTOR{next}.
% In order to initialize our $\HEAP$ array, a constant array \lstBTOR{state} is introduced and set according to the example's memory map given in Listing \ref{lst:mmap:init.mmap}.
In order to initialize our $\HEAP$ array with the contents of this example's memory map given in Listing \ref{lst:mmap:init.mmap}, an uninitialized array state is declared and the corresponding elements written accordingly.
% By omitting it's transition function, it is treated as a constant.
By omitting its transition function, it is treated as a primary input only used during initialization and can safely be ignored in later steps.
\begin{lstlisting}[style=btor2]
10 state 3        `\lstCMT{mmap}`
11 write 3 10 6 6 `\lstCMT{$\texttt{mmap}[0] \leftarrow 0$}`
12 write 3 11 7 6 `\lstCMT{$\texttt{mmap}[1] \leftarrow 0$}`
\end{lstlisting}

\noindent
% Next, all states are declared:
Since our actual machine states are used in the definition of constraints and transition functions, they must be declared in advance.
% The encoding continues with the declaration of machine states
\bigbreak
\noindent
$\bullet \;$ Accumulator registers: % $\ACCU_t$:
\begin{lstlisting}[style=btor2]
13 state 2 `\lstCMT{$\ACCU_0$}`
14 state 2 `\lstCMT{$\ACCU_1$}`
15 state 2 `\lstCMT{$\ACCU_2$}`
\end{lstlisting}

\noindent
$\bullet \;$ CAS memory registers: % $\MEM_t$:
\begin{lstlisting}[style=btor2]
16 state 2 `\lstCMT{$\MEM_0$}`
17 state 2 `\lstCMT{$\MEM_1$}`
18 state 2 `\lstCMT{$\MEM_2$}`
\end{lstlisting}

\noindent
$\bullet \;$ Store buffer address registers: % $\SBADR_t$:
\begin{lstlisting}[style=btor2]
19 state 2 `\lstCMT{$\SBADR_0$}`
20 state 2 `\lstCMT{$\SBADR_1$}`
21 state 2 `\lstCMT{$\SBADR_2$}`
\end{lstlisting}

\noindent
$\bullet \;$ Store buffer value registers: % $\SBVAL_t$:
\begin{lstlisting}[style=btor2]
22 state 2 `\lstCMT{$\SBVAL_0$}`
23 state 2 `\lstCMT{$\SBVAL_1$}`
24 state 2 `\lstCMT{$\SBVAL_2$}`
\end{lstlisting}

\noindent
$\bullet \;$ Store buffer full flags: % $\SBFULL_t$:
\begin{lstlisting}[style=btor2]
25 state 1 `\lstCMT{$\SBFULL_0$}`
26 state 1 `\lstCMT{$\SBFULL_1$}`
27 state 1 `\lstCMT{$\SBFULL_2$}`
\end{lstlisting}

\noindent
$\bullet \;$ Statement activation flags: % $\STMT_{t, pc}$:
\begin{lstlisting}[style=btor2]
28 state 1 `\lstCMT{$\STMT_{0, 0}$}`
29 state 1 `\lstCMT{$\STMT_{0, 1}$}`
30 state 1 `\lstCMT{$\STMT_{0, 2}$}`
31 state 1 `\lstCMT{$\STMT_{0, 3}$}`
32 state 1 `\lstCMT{$\STMT_{0, 4}$}`

33 state 1 `\lstCMT{$\STMT_{1, 0}$}`
34 state 1 `\lstCMT{$\STMT_{1, 1}$}`
35 state 1 `\lstCMT{$\STMT_{1, 2}$}`
36 state 1 `\lstCMT{$\STMT_{1, 3}$}`
37 state 1 `\lstCMT{$\STMT_{1, 4}$}`

38 state 1 `\lstCMT{$\STMT_{2, 0}$}`
39 state 1 `\lstCMT{$\STMT_{2, 1}$}`
40 state 1 `\lstCMT{$\STMT_{2, 2}$}`
41 state 1 `\lstCMT{$\STMT_{2, 3}$}`
42 state 1 `\lstCMT{$\STMT_{2, 4}$}`
43 state 1 `\lstCMT{$\STMT_{2, 5}$}`
\end{lstlisting}

\noindent
$\bullet \;$ Block flags: % $\BLOCK_{id, t}$:
\begin{lstlisting}[style=btor2]
44 state 1 `\lstCMT{$\BLOCK_{0, 0}$}`
45 state 1 `\lstCMT{$\BLOCK_{0, 1}$}`
46 state 1 `\lstCMT{$\BLOCK_{0, 2}$}`
\end{lstlisting}

\noindent
$\bullet \;$ Halt flags: % $\HALT_t$:
\begin{lstlisting}[style=btor2]
47 state 1 `\lstCMT{$\HALT_0$}`
48 state 1 `\lstCMT{$\HALT_1$}`
49 state 1 `\lstCMT{$\HALT_2$}`
\end{lstlisting}

\noindent
$\bullet \;$ Shared memory: % $\HEAP$:
\begin{lstlisting}[style=btor2]
50 state 3 `\lstCMT{$\HEAP$}`
\end{lstlisting}

\noindent
$\bullet \;$ Exit flag: % $\EXIT$:
\begin{lstlisting}[style=btor2]
51 state 1 `\lstCMT{$\EXIT$}`
\end{lstlisting}

\noindent
$\bullet \;$ Exit code: % $\EXITCODE$:
\begin{lstlisting}[style=btor2]
52 state 2 `\lstCMT{$\EXITCODE$}`
\end{lstlisting}

\noindent
Same applies to the free transition variables $\THREAD_t$ and $\FLUSH_t$ serving as the model's input.
% serving as the model's inputs declared as free variables through the keyword \lstBTOR{input}.
% Because they serve as the model's input, we declare them as free variables through the keyword \lstBTOR{input}.
\begin{lstlisting}[style=btor2]
53 input 1 `\lstCMT{$\THREAD_0$}`
54 input 1 `\lstCMT{$\THREAD_1$}`
55 input 1 `\lstCMT{$\THREAD_2$}`

56 input 1 `\lstCMT{$\FLUSH_0$}`
57 input 1 `\lstCMT{$\FLUSH_1$}`
58 input 1 `\lstCMT{$\FLUSH_2$}`
\end{lstlisting}

\noindent
% In contrast to our {\SMTLIB} encoding, no additional declarations of helper variables are needed and our execution variables $\EXEC_{t, pc}$ can be defined by simple \lstBTOR{and} nodes.
In contrast to the {\SMTLIB} encoding, no additional declarations are needed for the introduction of helper variables and our execution variables $\EXEC_{t, pc}$ therefore defined as simple \lstBTOR{and} nodes.
% Execution variable definitions.
\begin{lstlisting}[style=btor2]
59 and 1 28 53        `\lstCMT{$\STMT_{0, 0} \land \THREAD_0 \equiv \EXEC_{0, 0}$}`
60 and 1 29 53        `\lstCMT{$\STMT_{0, 1} \land \THREAD_0 \equiv \EXEC_{0, 1}$}`
61 and 1 30 53        `\lstCMT{$\STMT_{0, 2} \land \THREAD_0 \equiv \EXEC_{0, 2}$}`
62 and 1 31 53        `\lstCMT{$\STMT_{0, 3} \land \THREAD_0 \equiv \EXEC_{0, 3}$}`
63 and 1 32 53        `\lstCMT{$\STMT_{0, 4} \land \THREAD_0 \equiv \EXEC_{0, 4}$}`

64 and 1 33 54        `\lstCMT{$\STMT_{1, 0} \land \THREAD_1 \equiv \EXEC_{1, 0}$}`
65 and 1 34 54        `\lstCMT{$\STMT_{1, 1} \land \THREAD_1 \equiv \EXEC_{1, 1}$}`
66 and 1 35 54        `\lstCMT{$\STMT_{1, 2} \land \THREAD_1 \equiv \EXEC_{1, 2}$}`
67 and 1 36 54        `\lstCMT{$\STMT_{1, 3} \land \THREAD_1 \equiv \EXEC_{1, 3}$}`
68 and 1 37 54        `\lstCMT{$\STMT_{1, 4} \land \THREAD_1 \equiv \EXEC_{1, 4}$}`

69 and 1 38 55        `\lstCMT{$\STMT_{2, 0} \land \THREAD_2 \equiv \EXEC_{2, 0}$}`
70 and 1 39 55        `\lstCMT{$\STMT_{2, 1} \land \THREAD_2 \equiv \EXEC_{2, 1}$}`
71 and 1 40 55        `\lstCMT{$\STMT_{2, 2} \land \THREAD_2 \equiv \EXEC_{2, 2}$}`
72 and 1 41 55        `\lstCMT{$\STMT_{2, 3} \land \THREAD_2 \equiv \EXEC_{2, 3}$}`
73 and 1 42 55        `\lstCMT{$\STMT_{2, 4} \land \THREAD_2 \equiv \EXEC_{2, 4}$}`
74 and 1 43 55        `\lstCMT{$\STMT_{2, 5} \land \THREAD_2 \equiv \EXEC_{2, 5}$}`
\end{lstlisting}

\noindent
% Checkpoint variable definitions.
% Similarly, the checkpoint variable $\CHECK_0$ is defined through a conjunction over the corresponding block flags $\BLOCK_{0, t}$.
% Due to the fixed arity of {\BTOR} nodes,
The fixed arity of {\BTOR} operators imposes a minor inconvenience
% The absence of variadic versions for associative operators in {\BTOR} imposes a minor inconvenience
% when building expressions over a larger number of variables.
% if a connective has to be applied over a large number of variables.
if the same connective has to be applied over a larger number of variables.
% applying a connective over a larger number of variables.
% as it requires a cascade of \lstBTOR{and} nodes to define the checkpoint variable $\CHECK_0$ as a conjunction over the corresponding block flags $\BLOCK_{0, t}$.
For example, a cascade of \lstBTOR{and} nodes is required to build the conjunction over all block flags $\BLOCK_{0, t}$ representing the checkpoint variable $\CHECK_0$.
% For example, the conjunction over all block flags $\BLOCK_{0, t}$ representing the checkpoint variable $\CHECK_0$ requires a cascade of \lstBTOR{and} nodes.
% $ \CHECK_0 \equiv \bigwedge_{0 \leq t \leq 3} \BLOCK_{0, t}$.
\begin{lstlisting}[style=btor2]
75 and 1 44 45        `\lstCMT{$\BLOCK_{0, 0} \land \BLOCK_{0, 1}$}`
76 and 1 46 75        `\lstCMT{$\LPREV \land \BLOCK_{0, 2} \equiv \CHECK_0$}`
\end{lstlisting}

\noindent
% Cardinality constraints.
% Similarly, the clause representing the required greater than one constraint
% Similarly, the disjunction serving as the \emph{greater than zero} part of our \emph{exactly one} cardinality constraint is defined by a cascade of \lstBTOR{or} nodes.
Similarly, the disjunction serving as the \emph{at-least-one} predicate of our \emph{exactly-one} constraint is defined by a cascade of \lstBTOR{or} nodes
% $>^1_7\!(\THREAD_0, \THREAD_1, \THREAD_2, \FLUSH_0, \FLUSH_1, \FLUSH_2, \EXIT)$
% \[
  % \EXIT \lor (\bigvee_{0 \leq t \leq 3} \THREAD_t \lor \FLUSH_t)
% \]
% needs to be defined through a cascade of disjunctions.
\begin{lstlisting}[style=btor2]
77 or 1 53 54         `\lstCMT{$\THREAD_0 \lor \THREAD_1$}`
78 or 1 55 77         `\lstCMT{$\LPREV \lor \THREAD_2$}`
79 or 1 56 78         `\lstCMT{$\LPREV \lor \FLUSH_0$}`
80 or 1 57 79         `\lstCMT{$\LPREV \lor \FLUSH_1$}`
81 or 1 58 80         `\lstCMT{$\LPREV \lor \FLUSH_2$}`
82 or 1 51 81         `\lstCMT{$\LPREV \lor \EXIT$}`
\end{lstlisting}
% The corresponding invariant is then added through the keyword \lstBTOR{constraint}.
% and the corresponding invariant then added through the keyword \lstBTOR{constraint}.
and added as an invariant through the keyword \lstBTOR{constraint}.
% `\lstCMT{$>^1_7\!(\THREAD_0, \ldots, \FLUSH_0, \ldots, \EXIT)$}`
\begin{lstlisting}[style=btor2]
83 constraint 82
\end{lstlisting}
% Sequential counter constraint
% Definition of the \emph{at-most-one} predicate, implemented by Carsten Sinz's sequential counter constraint $\text{LT}^{7, 1}_{\text{SEQ}}$ \cite{ref:Sinz} starts with the declaration of the required auxiliary input variables.
Definition of the remaining \emph{at-most-one} predicate starts with the declaration of auxiliary input variables required by Carsten Sinz's sequential counter constraint $\text{LT}^{7, 1}_{\text{SEQ}}$ \cite{ref:Sinz},
% The \emph{exactly-one} constraint is then completed by adding the \emph{at-most-one} predicate implemented through Carsten Sinz's sequential counter constraint $\text{LT}^{7, 1}_{\text{SEQ}}$.
\begin{lstlisting}[style=btor2]
84 input 1            `\lstCMT{$\THREAD^{\text{aux}}_0$}`
85 input 1            `\lstCMT{$\THREAD^{\text{aux}}_1$}`
86 input 1            `\lstCMT{$\THREAD^{\text{aux}}_2$}`
87 input 1            `\lstCMT{$\FLUSH^{\text{aux}}_0$}`
88 input 1            `\lstCMT{$\FLUSH^{\text{aux}}_1$}`
89 input 1            `\lstCMT{$\FLUSH^{\text{aux}}_2$}`
\end{lstlisting}
% followed by the definition of clauses.
followed by the corresponding set of clauses.
\begin{lstlisting}[style=btor2]
90 or 1 -53 84        `\lstCMT{$\lnot \THREAD_0 \lor \THREAD_0^{\text{aux}}$}`
91 or 1 -54 85        `\lstCMT{$\lnot \THREAD_1 \lor \THREAD_1^{\text{aux}}$}`
92 or 1 -84 85        `\lstCMT{$\lnot \THREAD_0^{\text{aux}} \lor \THREAD_1^{\text{aux}}$}`
93 or 1 -54 -84       `\lstCMT{$\lnot \THREAD_1 \lor \lnot \THREAD_0^{\text{aux}}$}`
94 or 1 -55 86        `\lstCMT{$\lnot \THREAD_2 \lor \THREAD_2^{\text{aux}}$}`
95 or 1 -85 86        `\lstCMT{$\lnot \THREAD_1^{\text{aux}} \lor \THREAD_2^{\text{aux}}$}`
96 or 1 -55 -85       `\lstCMT{$\lnot \THREAD_2 \lor \lnot \THREAD_1^{\text{aux}}$}`
97 or 1 -56 87        `\lstCMT{$\lnot \FLUSH_0 \lor \FLUSH_0^{\text{aux}}$}`
98 or 1 -86 87        `\lstCMT{$\lnot \THREAD_2^{\text{aux}} \lor \FLUSH_0^{\text{aux}}$}`
99 or 1 -56 -86       `\lstCMT{$\lnot \FLUSH_0 \lor \THREAD_2^{\text{aux}}$}`
100 or 1 -57 88       `\lstCMT{$\lnot \FLUSH_1 \lor \FLUSH_0^{\text{aux}}$}`
101 or 1 -87 88       `\lstCMT{$\lnot \FLUSH_0^{\text{aux}} \lor \FLUSH_1^{\text{aux}}$}`
102 or 1 -57 -87      `\lstCMT{$\lnot \FLUSH_1 \lor \lnot \FLUSH_0^{\text{aux}}$}`
103 or 1 -58 89       `\lstCMT{$\lnot \FLUSH_2 \lor \FLUSH_1^{\text{aux}}$}`
104 or 1 -88 89       `\lstCMT{$\lnot \FLUSH_1^{\text{aux}} \lor \FLUSH_2^{\text{aux}}$}`
105 or 1 -58 -88      `\lstCMT{$\lnot \FLUSH_2 \lor \lnot \FLUSH_1^{\text{aux}}$}`
106 or 1 -51 -89      `\lstCMT{$\lnot \EXIT \lor \lnot \FLUSH_2^{\text{aux}}$}`
\end{lstlisting}
% The final $\leq^1_7$ \lstBTOR{constraint} can then be built by a conjunction over the previously defined clauses.
The \emph{exactly-one} constraint can now be completed by building a conjunction over aforementioned clauses
\begin{lstlisting}[style=btor2]
107 and 1 90 91       `\lstCMT{$(\lnot \THREAD_0 \lor \THREAD_0^{\text{aux}}) \land (\lnot \THREAD_1 \lor \THREAD_1^{\text{aux}})$}`
108 and 1 92 107      `\lstCMT{$\LPREV \land (\lnot \THREAD_0^{\text{aux}} \lor \THREAD_1^{\text{aux}})$}`
109 and 1 93 108      `\lstCMT{$\LPREV \land (\lnot \THREAD_1 \lor \lnot \THREAD_0^{\text{aux}})$}`
110 and 1 94 109      `\lstCMT{$\LPREV \land (\lnot \THREAD_2 \lor \THREAD_2^{\text{aux}})$}`
111 and 1 95 110      `\lstCMT{$\LPREV \land (\lnot \THREAD_1^{\text{aux}} \lor \THREAD_2^{\text{aux}})$}`
112 and 1 96 111      `\lstCMT{$\LPREV \land (\lnot \THREAD_2 \lor \lnot \THREAD_1^{\text{aux}})$}`
113 and 1 97 112      `\lstCMT{$\LPREV \land (\lnot \FLUSH_0 \lor \FLUSH_0^{\text{aux}})$}`
114 and 1 98 113      `\lstCMT{$\LPREV \land (\lnot \THREAD_2^{\text{aux}} \lor \FLUSH_0^{\text{aux}})$}`
115 and 1 99 114      `\lstCMT{$\LPREV \land (\lnot \FLUSH_0 \lor \THREAD_2^{\text{aux}})$}`
116 and 1 100 115     `\lstCMT{$\LPREV \land (\lnot \FLUSH_1 \lor \FLUSH_0^{\text{aux}})$}`
117 and 1 101 116     `\lstCMT{$\LPREV \land (\lnot \FLUSH_0^{\text{aux}} \lor \FLUSH_1^{\text{aux}})$}`
118 and 1 102 117     `\lstCMT{$\LPREV \land (\lnot \FLUSH_1 \lor \lnot \FLUSH_0^{\text{aux}})$}`
119 and 1 103 118     `\lstCMT{$\LPREV \land (\lnot \FLUSH_2 \lor \FLUSH_1^{\text{aux}})$}`
120 and 1 104 119     `\lstCMT{$\LPREV \land (\lnot \FLUSH_1^{\text{aux}} \lor \FLUSH_2^{\text{aux}})$}`
121 and 1 105 120     `\lstCMT{$\LPREV \land (\lnot \FLUSH_2 \lor \lnot \FLUSH_1^{\text{aux}})$}`
122 and 1 106 121     `\lstCMT{$\LPREV \land (\lnot \EXIT \lor \lnot \FLUSH_2^{\text{aux}})$}`
\end{lstlisting}
and adding yet another invariant.
\begin{lstlisting}[style=btor2]
123 constraint 122
\end{lstlisting}
% `\lstCMT{$\leq^1_7\!(\THREAD_0, \ldots, \FLUSH_0, \ldots, \EXIT)$}`

\noindent
% Store buffer constraints.
% The other store buffer,
% The remaining scheduling constraints can be defined rather effortlessly.
% \bigbreak
% \noindent
% $\bullet \;$ Disable flushing empty store buffer and executing barrier instructions while it is full:
% Transitions are then further restricted by the remaining scheduling constraints like
% The number of valid transitions is then further restricted by our remaining scheduling constraints like
This cardinality constraint is then again further influenced
by prohibiting certain transitions like
% Possible transitions are then further restricted by our common set of scheduling constraints like
% Again, execution is further restricted,
% Next, we also introduce our common scheduling constraints like
flushing an empty store buffer or executing barrier instructions if it is full,
\begin{lstlisting}[style=btor2]
124 or 1 29 32        `\lstCMT{$\STMT_{0, 1} \lor \STMT_{0, 4}$}`
125 implies 1 124 -53 `\lstCMT{$\LPREV \implies \lnot \THREAD_0$}`
126 ite 1 25 125 -56  `\lstCMT{$\ITE(\SBFULL_0, \;\LPREV\:, \lnot \FLUSH_0)$}`
127 constraint 126

128 or 1 34 37        `\lstCMT{$\STMT_{1, 1} \lor \STMT_{1, 4}$}`
129 implies 1 128 -54 `\lstCMT{$\LPREV \implies \lnot \THREAD_1$}`
130 ite 1 26 129 -57  `\lstCMT{$\ITE(\SBFULL_1, \;\LPREV\:, \lnot \FLUSH_1)$}`
131 constraint 130

132 implies 1 58 27   `\lstCMT{$\FLUSH_2 \implies \SBFULL_2$}`
133 constraint 132
\end{lstlisting}

\noindent
% Checkpoint constraints.
executing a thread while it is waiting for a checkpoint,
\begin{lstlisting}[style=btor2]
134 and 1 44 -76      `\lstCMT{$\BLOCK_{0, 0} \land \lnot \CHECK_0$}`
135 implies 1 134 -53 `\lstCMT{$\LPREV \implies \lnot \THREAD_0$}`
136 constraint 135

137 and 1 45 -76      `\lstCMT{$\BLOCK_{0, 1} \land \lnot \CHECK_0$}`
138 implies 1 137 -54 `\lstCMT{$\LPREV \implies \lnot \THREAD_1$}`
139 constraint 138

140 and 1 46 -76      `\lstCMT{$\BLOCK_{0, 2} \land \lnot \CHECK_0$}`
141 implies 1 140 -55 `\lstCMT{$\LPREV \implies \lnot \THREAD_2$}`
142 constraint 141
\end{lstlisting}

\noindent
% Halt constraints.
and stopping halted threads from being scheduled.
% and halt related constraints ...
\begin{lstlisting}[style=btor2]
143 implies 1 47 -53  `\lstCMT{$\HALT_0 \implies \lnot \THREAD_0$}`
144 constraint 143

145 implies 1 48 -54  `\lstCMT{$\HALT_1 \implies \lnot \THREAD_1$}`
146 constraint 145

147 implies 1 49 -55  `\lstCMT{$\HALT_2 \implies \lnot \THREAD_2$}`
148 constraint 147
\end{lstlisting}

\noindent
% Accumulator state definitions.
As mentioned earlier, the main advantage of the {\BTOR} format lies in the ability to define states in combination with their transition function.
% In case of our machine states therefore follows a pretty basic pattern:
In our case, definition of machine states therefore follows a pretty basic pattern:
% Our machine state can therefore be defined by following a pretty basic pattern:
% This feature allows us to define our machine states
% after initializing the state by using the keyword \lstBTOR{init},
after initializing the state with the keyword \lstBTOR{init},
% and specifying its successor through the
% and adding the set of nodes for defining its successor,
% its transition function is implemented by the according set of nodes and
% its transition function is defined through the appropriate set of nodes and registered with the corresponding keyword \lstBTOR{next}.
its successor is defined through the appropriate set of nodes and
the resulting expression
registered as the given state's transition function by using the corresponding keyword \lstBTOR{next}.
% and defining its successor through the appropriate set of nodes,
% and adding the set of nodes yielding its successor,
% and specified to be the specific states transition function through the keyword \lstBTOR{next}.
% the specific state's transition function can be registered with accompanying keyword \lstBTOR{next}.
% Due to the reuse of nodes, the encoding is much more compact and given below for completeness.
\bigbreak
\noindent
$\bullet\ $ Accumulator registers:
\begin{lstlisting}[style=btor2]
149 init 2 13 6       `\lstCMT{$\ACCU^0_0 \leftarrow 0$}`
150 add 2 13 7        `\lstCMT{$\ACCU_0 + 1$}`
151 ite 2 59 150 13   `\lstCMT{$\ITE(\EXEC_{0, 0}, \,\LPREV\,, \ACCU_0) \equiv \texttt{ADDI}_0$}`
152 read 2 50 7       `\lstCMT{$\READ(1)$}`
153 eq 1 19 7         `\lstCMT{$\SBADR_0 = 1$}`
154 and 1 25 153      `\lstCMT{$\LPREV \land \SBFULL_0$}`
155 ite 2 154 22 152  `\lstCMT{$\ITE(\,\LPREV\,, \SBVAL_0, \READ(1))$}`
156 ite 2 61 155 151  `\lstCMT{$\ITE(\EXEC_{0, 2}, \,\LPREV\,, \, \texttt{ADDI}_0)$}`
157 next 2 13 156     `\lstCMT{$\ACCU'_0 \leftarrow \RPREV$}`

158 init 2 14 6       `\lstCMT{$\ACCU^0_1 \leftarrow 0$}`
159 add 2 14 7        `\lstCMT{$\ACCU_1 + 1$}`
160 ite 2 64 159 14   `\lstCMT{$\ITE(\EXEC_{1, 0}, \,\LPREV\,, \ACCU_1) \equiv \texttt{ADDI}_1$}`
161 read 2 50 6       `\lstCMT{$\READ(0)$}`
162 eq 1 20 6         `\lstCMT{$\SBADR_1 = 0$}`
163 and 1 26 162      `\lstCMT{$\LPREV \land \SBFULL_1$}`
164 ite 2 163 23 161  `\lstCMT{$\ITE(\,\LPREV\,, \SBVAL_1, \READ(0))$}`
165 ite 2 66 164 160  `\lstCMT{$\ITE(\EXEC_{1, 2}, \,\LPREV\,, \, \texttt{ADDI}_1)$}`
166 next 2 14 165     `\lstCMT{$\ACCU'_1 \leftarrow \RPREV$}`

167 init 2 15 6       `\lstCMT{$\ACCU^0_2 \leftarrow 0$}`
168 eq 1 21 6         `\lstCMT{$\SBADR_2 = 0$}`
169 and 1 27 168      `\lstCMT{$\LPREV \land \SBFULL_2$}`
170 ite 2 169 24 161  `\lstCMT{$\ITE(\,\LPREV\,, \SBVAL_2, \READ(0))$}`
171 add 2 15 170      `\lstCMT{$\LPREV + \ACCU_2$}`
172 ite 2 70 171 15   `\lstCMT{$\ITE(\EXEC_{2, 1}, \,\LPREV\,, \ACCU_2) \equiv \texttt{ADD}$}`
173 eq 1 21 7         `\lstCMT{$\SBADR_2 = 1$}`
174 and 1 27 173      `\lstCMT{$\LPREV \land \SBFULL_2$}`
175 ite 2 174 24 152  `\lstCMT{$\ITE(\,\LPREV\,, \SBVAL_2, \READ(1))$}`
176 add 2 15 175      `\lstCMT{$\LPREV + \ACCU_2$}`
177 ite 2 71 176 172  `\lstCMT{$\ITE(\EXEC_{2, 2}, \,\LPREV\,, \texttt{ADD})$}`
178 next 2 15 177     `\lstCMT{$\ACCU'_2 \leftarrow \RPREV$}`
\end{lstlisting}
                      % `\lstCMT{$\ITE(\EXEC_{2, 2}, \,\LPREV\,, \ITE(\EXEC_{2, 1}, \ITE(\,\LPREV\,, \SBVAL_2, \HEAP[0]), \ACCU_2))$}`
                      % `\lstCMT{\hspace{0.8cm}$\LPREV,$}`
                      % `\lstCMT{\hspace{0.8cm}$\ITE(\EXEC_{2, 1},$}`

\noindent
$\bullet\ $ CAS memory registers:
\begin{lstlisting}[style=btor2]
179 init 2 16 6       `\lstCMT{$\MEM^0_0 \leftarrow 0$}`
180 next 2 16 16      `\lstCMT{$\MEM'_0 \leftarrow \MEM_0$}`

181 init 2 17 6       `\lstCMT{$\MEM^0_1 \leftarrow 0$}`
182 next 2 17 17      `\lstCMT{$\MEM'_1 \leftarrow \MEM_1$}`

183 init 2 18 6       `\lstCMT{$\MEM^0_2 \leftarrow 0$}`
184 next 2 18 18      `\lstCMT{$\MEM'_2 \leftarrow \MEM_2$}`
\end{lstlisting}

\noindent
$\bullet\ $ Store buffer address registers:
\begin{lstlisting}[style=btor2]
185 init 2 19 6       `\lstCMT{$\SBADR^0_0 \leftarrow 0$}`
186 ite 2 60 6 19     `\lstCMT{$\ITE(\EXEC_{0, 1}, 0, \SBADR_0)$}`
187 next 2 19 186     `\lstCMT{$\SBADR'_0 \leftarrow \RPREV$}`

188 init 2 20 6       `\lstCMT{$\SBADR^0_1 \leftarrow 0$}`
189 ite 2 65 7 20     `\lstCMT{$\ITE(\EXEC_{1, 1}, 0, \SBADR_1)$}`
190 next 2 20 189     `\lstCMT{$\SBADR'_1 \leftarrow \RPREV$}`

191 init 2 21 6       `\lstCMT{$\SBADR^0_2 \leftarrow 0$}`
192 next 2 21 21      `\lstCMT{$\SBADR'_2 \leftarrow \SBADR_2$}`
\end{lstlisting}

\noindent
$\bullet\ $ Store buffer value registers:
\begin{lstlisting}[style=btor2]
193 init 2 22 6       `\lstCMT{$\SBVAL^0_0 \leftarrow 0$}`
194 ite 2 60 13 22    `\lstCMT{$\ITE(\EXEC_{0, 1}, \ACCU_0, \SBVAL_0)$}`
195 next 2 22 194     `\lstCMT{$\SBVAL'_0 \leftarrow \RPREV$}`

196 init 2 23 6       `\lstCMT{$\SBVAL^0_1 \leftarrow 0$}`
197 ite 2 65 14 23    `\lstCMT{$\ITE(\EXEC_{1, 1}, \ACCU_1, \SBVAL_1)$}`
198 next 2 23 197     `\lstCMT{$\SBVAL'_1 \leftarrow \RPREV$}`

199 init 2 24 6       `\lstCMT{$\SBVAL^0_1 \leftarrow 0$}`
200 next 2 24 24      `\lstCMT{$\SBVAL'_2 \leftarrow \SBVAL_2$}`
\end{lstlisting}

\noindent
$\bullet\ $ Store buffer full flags:
\begin{lstlisting}[style=btor2]
201 init 1 25 4       `\lstCMT{$\SBFULL^0_0 \leftarrow 0$}`
202 or 1 60 25        `\lstCMT{$\EXEC_{0, 1} \lor \SBFULL_0$}`
203 ite 1 56 4 202    `\lstCMT{$\ITE(\FLUSH_0, 0, \,\LPREV\,)$}`
204 next 1 25 203     `\lstCMT{$\SBFULL'_0 \leftarrow \RPREV$}`

205 init 1 26 4       `\lstCMT{$\SBFULL^0_1 \leftarrow 0$}`
206 or 1 65 26        `\lstCMT{$\EXEC_{1, 1} \lor \SBFULL_1$}`
207 ite 1 57 4 206    `\lstCMT{$\ITE(\FLUSH_1, 0, \,\LPREV\,)$}`
208 next 1 26 207     `\lstCMT{$\SBFULL'_1 \leftarrow \RPREV$}`

209 init 1 27 4       `\lstCMT{$\SBFULL^0_2 \leftarrow 0$}`
210 ite 1 58 4 27     `\lstCMT{$\ITE(\FLUSH_2, 0, \SBFULL_2)$}`
211 next 1 27 210     `\lstCMT{$\SBFULL'_2 \leftarrow \RPREV$}`
\end{lstlisting}

\noindent
$\bullet\ $ Statement activation flags:
\begin{lstlisting}[style=btor2]
212 init 1 28 5       `\lstCMT{$\STMT^0_{0, 0} \leftarrow 1$}`
213 and 1 28 -59      `\lstCMT{$\STMT_{0, 0} \land \lnot \EXEC_{0, 0}$}`
214 next 1 28 213     `\lstCMT{$\STMT'_{0, 0} \leftarrow \RPREV$}`

215 init 1 29 4       `\lstCMT{$\STMT^0_{0, 1} \leftarrow 0$}`
216 and 1 29 -60      `\lstCMT{$\STMT_{0, 1} \land \lnot \EXEC_{0, 1}$}`
217 ite 1 28 59 216   `\lstCMT{$\ITE(\STMT_{0, 0}, \EXEC_{0, 0}, \,\RPREV\,)$}`
218 next 1 29 217     `\lstCMT{$\STMT'_{0, 1} \leftarrow \RPREV$}`

219 init 1 30 4       `\lstCMT{$\STMT^0_{0, 2} \leftarrow 0$}`
220 and 1 30 -61      `\lstCMT{$\STMT_{0, 2} \land \lnot \EXEC_{0, 2}$}`
221 ite 1 29 60 220   `\lstCMT{$\ITE(\STMT_{0, 1}, \EXEC_{0, 1}, \,\RPREV\,)$}`
222 next 1 30 221     `\lstCMT{$\STMT'_{0, 2} \leftarrow \RPREV$}`

223 init 1 31 4       `\lstCMT{$\STMT^0_{0, 3} \leftarrow 0$}`
224 and 1 31 -62      `\lstCMT{$\STMT_{0, 3} \land \lnot \EXEC_{0, 3}$}`
225 ite 1 30 61 224   `\lstCMT{$\ITE(\STMT_{0, 2}, \EXEC_{0, 2}, \,\RPREV\,)$}`
226 next 1 31 225     `\lstCMT{$\STMT'_{0, 3} \leftarrow \RPREV$}`

227 init 1 32 4       `\lstCMT{$\STMT^0_{0, 4} \leftarrow 0$}`
228 and 1 32 -63      `\lstCMT{$\STMT_{0, 4} \land \lnot \EXEC_{0, 4}$}`
229 ite 1 31 62       `\lstCMT{$\ITE(\STMT_{0, 3}, \EXEC_{0, 3}, \,\RPREV\,)$}`
230 next 1 32 229     `\lstCMT{$\STMT'_{0, 4} \leftarrow \RPREV$}`

231 init 1 33 5       `\lstCMT{$\STMT^0_{1, 0} \leftarrow 1$}`
232 and 1 33 -64      `\lstCMT{$\STMT_{1, 0} \land \lnot \EXEC_{1, 0}$}`
233 next 1 33 232     `\lstCMT{$\STMT'_{1, 0} \leftarrow \RPREV$}`

234 init 1 34 4       `\lstCMT{$\STMT^0_{1, 1} \leftarrow 0$}`
235 and 1 34 -65      `\lstCMT{$\STMT_{1, 1} \land \lnot \EXEC_{1, 1}$}`
236 ite 1 33 64 235   `\lstCMT{$\ITE(\STMT_{1, 0}, \EXEC_{1, 0}, \,\RPREV\,)$}`
237 next 1 34 236     `\lstCMT{$\STMT'_{1, 1} \leftarrow \RPREV$}`

238 init 1 35 4       `\lstCMT{$\STMT^0_{1, 2} \leftarrow 0$}`
239 and 1 35 -66      `\lstCMT{$\STMT_{1, 2} \land \lnot \EXEC_{1, 2}$}`
240 ite 1 34 65 239   `\lstCMT{$\ITE(\STMT_{1, 1}, \EXEC_{1, 1}, \,\RPREV\,)$}`
241 next 1 35 240     `\lstCMT{$\STMT'_{1, 2} \leftarrow \RPREV$}`

242 init 1 36 4       `\lstCMT{$\STMT^0_{1, 3} \leftarrow 0$}`
243 and 1 36 -67      `\lstCMT{$\STMT_{1, 3} \land \lnot \EXEC_{1, 3}$}`
244 ite 1 35 66 243   `\lstCMT{$\ITE(\STMT_{1, 2}, \EXEC_{1, 2}, \,\RPREV\,)$}`
245 next 1 36 244     `\lstCMT{$\STMT'_{1, 3} \leftarrow \RPREV$}`

246 init 1 37 4       `\lstCMT{$\STMT^0_{1, 4} \leftarrow 0$}`
247 and 1 37 -68      `\lstCMT{$\STMT_{1, 4} \land \lnot \EXEC_{0, 4}$}`
248 ite 1 36 67 247   `\lstCMT{$\ITE(\STMT_{1, 3}, \EXEC_{1, 3}, \,\RPREV\,)$}`
249 next 1 37 248     `\lstCMT{$\STMT'_{1, 4} \leftarrow \RPREV$}`

250 init 1 38 5       `\lstCMT{$\STMT^0_{2, 0} \leftarrow 1$}`
251 and 1 38 -69      `\lstCMT{$\STMT_{2, 0} \land \lnot \EXEC_{2, 0}$}`
252 next 1 38 251     `\lstCMT{$\STMT'_{2, 0} \leftarrow \RPREV$}`

253 init 1 39 4       `\lstCMT{$\STMT^0_{2, 1} \leftarrow 0$}`
254 and 1 39 -70      `\lstCMT{$\STMT_{2, 1} \land \lnot \EXEC_{2, 1}$}`
255 ite 1 38 69 254   `\lstCMT{$\ITE(\STMT_{2, 0}, \EXEC_{2, 0}, \,\RPREV\,)$}`
256 next 1 39 255     `\lstCMT{$\STMT'_{2, 1} \leftarrow \RPREV$}`

257 init 1 40 4       `\lstCMT{$\STMT^0_{2, 2} \leftarrow 0$}`
258 and 1 40 -71      `\lstCMT{$\STMT_{2, 2} \land \lnot \EXEC_{2, 2}$}`
259 ite 1 39 70 258   `\lstCMT{$\ITE(\STMT_{2, 1}, \EXEC_{2, 1}, \,\RPREV\,)$}`
260 next 1 40 259     `\lstCMT{$\STMT'_{2, 2} \leftarrow \RPREV$}`

261 init 1 41 4       `\lstCMT{$\STMT^0_{2, 3} \leftarrow 0$}`
262 and 1 41 -72      `\lstCMT{$\STMT_{2, 3} \land \lnot \EXEC_{2, 3}$}`
263 ite 1 40 71 262   `\lstCMT{$\ITE(\STMT_{2, 2}, \EXEC_{2, 2}, \,\RPREV\,)$}`
264 next 1 41 263     `\lstCMT{$\STMT'_{2, 3} \leftarrow \RPREV$}`

265 init 1 42 4       `\lstCMT{$\STMT^0_{2, 4} \leftarrow 0$}`
266 and 1 42 -73      `\lstCMT{$\STMT_{2, 4} \land \lnot \EXEC_{2, 4}$}`
267 eq 1 15 6         `\lstCMT{$\ACCU_2 = 0$}`
268 and 1 72 -267     `\lstCMT{$\EXEC_{2, 3} \land \ACCU_2 \neq 0$}`
269 ite 1 41 268 266  `\lstCMT{$\ITE(\STMT_{2, 3}, \,\LPREV\,, \STMT_{2, 4} \land \lnot \EXEC_{2, 4})$}`
270 next 1 42 269     `\lstCMT{$\STMT'_{2, 4} \leftarrow \RPREV$}`

271 init 1 43 4       `\lstCMT{$\STMT^0_{2, 5} \leftarrow 0$}`
272 and 1 43 -74      `\lstCMT{$\STMT_{2, 5} \land \lnot \EXEC_{2, 5}$}`
273 and 1 72 267      `\lstCMT{$\EXEC_{2, 3} \land \ACCU_2 = 0$}`
274 ite 1 41 273 272  `\lstCMT{$\ITE(\STMT_{2, 3}, \,\LPREV\,, \STMT_{2, 5} \land \lnot \EXEC_{2, 5})$}`
275 next 1 43 274     `\lstCMT{$\STMT'_{2, 5} \leftarrow \RPREV$}`
\end{lstlisting}

\noindent
$\bullet\ $ Block flags:
\begin{lstlisting}[style=btor2]
276 init 1 44 4       `\lstCMT{$\BLOCK^0_{0, 0} \leftarrow 0$}`
277 or 1 62 44        `\lstCMT{$\EXEC_{0, 3} \lor \BLOCK_{0, 0}$}`
278 ite 1 76 4 277    `\lstCMT{$\ITE(\CHECK_0, 0, \,\RPREV\,)$}`
279 next 1 44 278     `\lstCMT{$\BLOCK'_{0, 0} \leftarrow \RPREV$}`

280 init 1 45 4       `\lstCMT{$\BLOCK^0_{0, 1} \leftarrow 0$}`
281 or 1 67 45        `\lstCMT{$\EXEC_{1, 3} \lor \BLOCK_{0, 1}$}`
282 ite 1 76 4 281    `\lstCMT{$\ITE(\CHECK_0, 0, \,\RPREV\,)$}`
283 next 1 45 282     `\lstCMT{$\BLOCK'_{0, 1} \leftarrow \RPREV$}`

284 init 1 46 4       `\lstCMT{$\BLOCK^0_{0, 2} \leftarrow 0$}`
285 or 1 69 46        `\lstCMT{$\EXEC_{2, 0} \lor \BLOCK_{0, 2}$}`
286 ite 1 76 4 285    `\lstCMT{$\ITE(\CHECK_0, 0, \,\RPREV\,)$}`
287 next 1 46 286     `\lstCMT{$\BLOCK'_{0, 2} \leftarrow \RPREV$}`
\end{lstlisting}

\noindent
$\bullet\ $ Halt flags:
\begin{lstlisting}[style=btor2]
288 init 1 47 4       `\lstCMT{$\HALT^0_0 \leftarrow 0$}`
289 or 1 63 47        `\lstCMT{$\EXEC_{0, 4} \lor \HALT_0$}`
290 next 1 47 289     `\lstCMT{$\HALT'_0 \leftarrow \RPREV$}`

291 init 1 48 4       `\lstCMT{$\HALT^0_1 \leftarrow 0$}`
292 or 1 68 48        `\lstCMT{$\EXEC_{1, 4} \lor \HALT_1$}`
293 next 1 48 292     `\lstCMT{$\HALT'_1 \leftarrow \RPREV$}`

294 init 1 49 4       `\lstCMT{$\HALT^0_2 \leftarrow 0$}`
295 next 1 49 49      `\lstCMT{$\HALT'_2 \leftarrow \HALT_2$}`
\end{lstlisting}

\noindent
$\bullet\ $ Shared memory:
\begin{lstlisting}[style=btor2]
296 init 3 50 12      `\lstCMT{$\HEAP^0 \leftarrow \texttt{mmap}$}`
297 write 3 50 19 22  `\lstCMT{$\WRITE(\SBADR_0, \SBVAL_0)$}`
298 ite 3 56 297 50   `\lstCMT{$\ITE(\FLUSH_0, \,\LPREV\,, \HEAP) \equiv \texttt{FLUSH}_0$}`
299 write 3 50 20 23  `\lstCMT{$\WRITE(\SBADR_1, \SBVAL_1)$}`
300 ite 3 57 299 298  `\lstCMT{$\ITE(\FLUSH_1, \,\LPREV\,, \texttt{FLUSH}_0) \equiv \texttt{FLUSH}_1$}`
301 write 3 50 21 24  `\lstCMT{$\WRITE(\SBADR_2, \SBVAL_2)$}`
302 ite 3 58 301 300  `\lstCMT{$\ITE(\FLUSH_2, \,\LPREV\,, \texttt{FLUSH}_1)$}`
303 next 3 50 302     `\lstCMT{$\HEAP' \leftarrow \RPREV$}`
\end{lstlisting}

\noindent
$\bullet\ $ Exit flag:
\begin{lstlisting}[style=btor2]
304 init 1 51 4       `\lstCMT{$\EXIT^0 \leftarrow 0$}`
305 and 1 289 292     `\lstCMT{$\HALT'_0 \land \HALT'_1$}`
306 and 1 49 305      `\lstCMT{$\LPREV \land \HALT_2$}`
307 or 1 51 306       `\lstCMT{$\LPREV \lor \EXIT$}`
308 or 1 73 307       `\lstCMT{$\LPREV \lor \EXEC_{2, 4}$}`
309 or 1 74 308       `\lstCMT{$\LPREV \lor \EXEC_{2, 5}$}`
310 next 1 51 309     `\lstCMT{$\EXIT' \leftarrow \RPREV$}`
\end{lstlisting}

\noindent
$\bullet\ $ Exit code:
\begin{lstlisting}[style=btor2]
311 init 2 52 6       `\lstCMT{$\EXITCODE^0 \leftarrow 0$}`
312 ite 2 73 6 52     `\lstCMT{$\ITE(\EXEC_{2, 4}, 0, \EXITCODE)$}`
313 ite 2 74 7 312    `\lstCMT{$\ITE(\EXEC_{2, 5}, 1, \,\RPREV\,)$}`
314 next 2 52 313     `\lstCMT{$\EXITCODE' \leftarrow \RPREV$}`
\end{lstlisting}

\noindent
Finally, we declare an exit code greater than zero as the only \lstBTOR{bad} state.
\begin{lstlisting}[style=btor2]
315 neq 1 6 52
316 bad 315
\end{lstlisting}

% \newpage
%
% \begin{minipage}[t]{0.45\textwidth}
% \begin{lstlisting}[style=btor2]
% 1 sort bitvec 1
% 2 sort bitvec 16
% 3 sort array 2 2
%
% 4 zero 1
% 5 one 1
%
% 6 zero 2
% 7 one 2
% 8 constd 2 5
% 9 constd 2 17
%
% 10 state 3 mmap
% 11 write 3 10 6 6
% 12 write 3 11 7 6
%
% 13 state 2 accu_0
% 14 state 2 accu_1
% 15 state 2 accu_2
%
% 16 state 2 mem_0
% 17 state 2 mem_1
% 18 state 2 mem_2
%
% 19 state 2 sb-adr_0
% 20 state 2 sb-adr_1
% 21 state 2 sb-adr_2
%
% 22 state 2 sb-val_0
% 23 state 2 sb-val_1
% 24 state 2 sb-val_2
%
% 25 state 1 sb-full_0
% 26 state 1 sb-full_1
% 27 state 1 sb-full_2
%
% 28 state 1 stmt_0_0
% 29 state 1 stmt_0_1
% 30 state 1 stmt_0_2
% 31 state 1 stmt_0_3
% 32 state 1 stmt_0_4
% \end{lstlisting}
% \end{minipage}
% \begin{minipage}[t]{0.45\textwidth}
% \begin{lstlisting}[style=btor2]
% 33 state 1 stmt_1_0
% 34 state 1 stmt_1_1
% 35 state 1 stmt_1_2
% 36 state 1 stmt_1_3
% 37 state 1 stmt_1_4
%
% 38 state 1 stmt_2_0
% 39 state 1 stmt_2_1
% 40 state 1 stmt_2_2
% 41 state 1 stmt_2_3
% 42 state 1 stmt_2_4
% 43 state 1 stmt_2_5
%
% 44 state 1 block_0_0
% 45 state 1 block_0_1
% 46 state 1 block_0_2
%
% 47 state 1 halt_0
% 48 state 1 halt_1
% 49 state 1 halt_2
%
% 50 state 3 heap
%
% 51 state 1 exit
%
% 52 state 2 exit-code
%
% 53 input 1 thread_0
% 54 input 1 thread_1
% 55 input 1 thread_2
%
% 56 input 1 flush_0
% 57 input 1 flush_1
% 58 input 1 flush_2
%
% 59 and 1 28 53 exec_0_0
% 60 and 1 29 53 exec_0_1
% 61 and 1 30 53 exec_0_2
% 62 and 1 31 53 exec_0_3
% 63 and 1 32 53 exec_0_4
% \end{lstlisting}
% \end{minipage}

% \subsubsection{BTOR2 generator functions}
% \todo[inline]{BTOR2 generator functions} %%%%%%%%%%%%%%%%%%%%%%%%%%%%%%%%%%%%%%%
%
% \begin{lstlisting}[style=c++]
% template <class ... T>
% std::string expr (const std::string & nid, const T & ... args)
% {
  % std::string e = nid;
  % (((e += ' ') += args), ...);
  % auto & end = e.back();
  % if (end == ' ')
    % end = eol;
  % else
    % e += eol;
  % return e;
% }
% \end{lstlisting}
%
% \todo[inline]{main encoding function (smtlib)} %%%%%%%%%%%%%%%%%%%%%%%%%%%%%%%%%
%
% \begin{lstlisting}[style=c++]
% void encode ()
% {
  % declare_sorts();
  % declare_constants();
  % define_mmap();
  % declare_states();
  % declare_inputs();
  % define_transitions();
  % define_constraints();
  % define_states();
% }
% \end{lstlisting}
%
% \todo[inline]{declare sorts} %%%%%%%%%%%%%%%%%%%%%%%%%%%%%%%%%%%%%%%%%%%%%%%%%%%
%
% \begin{lstlisting}[style=c++]
% void declare_sorts ()
% {
  % formula <<
    % sort_bitvec(sid_bool = nid(), "1") <<
    % sort_bitvec(sid_bv = nid(), std::to_string(word_size)) <<
    % sort_array(sid_heap = nid(), "2", "2") <<
    % eol;
% }
% \end{lstlisting}

