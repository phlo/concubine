\section{Testing and Debugging}

The basic functionality of our toolchain is validated by a total of 800 test cases.
While fixing {\CPP} related errors was relatively trivial due to the broad range of available tools,
debugging the generated SMT encodings
% on the other hand
% requires more sophisticated methods.
is a bit more tedious
since cause and effect have to be determined manually by inspecting rather comprehensive models and formulas.

In order to automatically validate the correctness of our encodings, we included a \texttt{replay} \CHANGE{mode} for reevaluating the execution sequence of a given trace by the virtual machine used for simulation.
% Therefore, we included a \texttt{replay} \CHANGE{mode} for reevaluating the execution sequence of a given trace by the virtual machine used for simulation in order to automatically validate the correctness of our encodings.
If a mismatch is found, it returns the first step in which the results start to differ and prints the corresponding states of both execution traces for comparison.
% prints the states of the first step in which the results start to differ and returns
% printing the first step in which the traces start to differ
% returning step in which the resulting models start to differ and print
This mode also simplified the encodings' development significantly,
% since the step in which the model starts to diverge from our machine model
% as it automatically finds the step in which the encoding starts to differ from our machine model and
since the condensed error trace helps to narrow down the potentially flawed SMT expressions.
% since it helps to narrow down the potentially erroneous SMT expressions by

Of course, only satisfiable formulas may be replayed.
% reevaluated by \texttt{replay}.
If the outcome of a valid execution sequence is accidentally unsatisfiable and therefore yields no trace, we still have to manually inspect the formula.
Unfortunately, no SMT solver supported the extraction of an \emph{unsatisfiable core} at the time we developed our encodings, but this feature starts to get implemented in recent solvers like Bitwuzla \cite{ref:Bitwuzla}.
However, without any additional constraints, the only reason our formulas can get unsatisfiable is due to a violation of the cardinality constraint and investigations may therefore be concentrated on expressions containing transition variables.

Another frequent cause for unintended results lies in the nature of bounded model checking.
Before expecting an error in the encoding, it is highly advisable to make sure that the targeted states are indeed reachable by the chosen bound.
For example, when a model for the \lstASM{CAS} variant of our concurrent counter experiments (described in the next section) was found during our initial trials, we immediately started to panic and questioned our encoding.
Since such situations already commonly occurred during development, we remembered the following proverb which emerged during that time and
soon realized that the problem indeed was yet another insufficient bound.
% later found the problem to be yet another insufficient bound.
% since we questioned our encoding which was previously believed to be corre
% only to find the problem being an insufficient bound.
% resulting in the following proverb:
\begin{center}
\itshape
If you ask yourself -- ``How can this be?'' -- maybe just the bound's too wee!
\end{center}
